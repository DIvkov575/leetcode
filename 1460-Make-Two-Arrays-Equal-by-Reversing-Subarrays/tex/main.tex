\documentclass{article}
\usepackage{amsmath} % Required for inserting images
\title{Conjecture about subarray reversals}
\author{Dmitriy Ivkov}
\date{August 2024}

\parindent 0pt
\begin{document}

    \maketitle

    \textbf{Conjecture:} Any permutation of a finite ordered collection of objects can be created by reversing sub-arrays of the source array \\

    \textbf{Setup} \\
    Let \( S := (s_1, s_2, \cdots, s_n) \) be a finite ordered collection. \\
    Let \(\pi(S)\) be a permutation of size \(n\) of the collection \(S\). \\
    A sub-array reversal \(R(\text{collection}, \text{start}, \text{stop})\) is defined by taking any subset of \(S\) and reversing the order of every element within the subset. \\
    Let \(S^{-1}(\text{elem})\) be a function to get the index of the first occurrence of an element in the collection \(S\). \\

    Consider a procedure \(F(S, \pi(S))\) on a collection \(S\) and \(\pi(S)\) which:
    \begin{enumerate}
        \item Let \( \text{stop} = \pi(S)^{-1}(s_1) \): finds the index of \(s_1\) within \(\pi(S)\).
        \item Performs \(R(\pi(S), 0, \text{stop})\): flips the subset from the first element to the first occurrence of the desired element.
    \end{enumerate}

    \textbf{Inductive Hypothesis} \\
    Assume that any permutation of a collection of size \(k\) can be created by reversing sub-arrays of the source array. That is, for any permutation \(\pi\) of a collection \(S\) of size \(k\), there exists a sequence of sub-array reversals that transforms \(S\) into \(\pi\).

    \textbf{Base Case} \\
    For \(k = 1\), there is only one element in the collection. The only permutation is the collection itself, which trivially satisfies the hypothesis because no sub-array reversals are needed.

    \textbf{Inductive Step} \\
    Consider a collection \(S\) of size \(k+1\) and a permutation \(\pi\) of \(S\). We need to show that \(\pi\) can be obtained from \(S\) using sub-array reversals.

    \begin{enumerate}
        \item Find the position of the first element \(s_1\) of \(S\) in \(\pi(S)\). Let \(\text{stop} = \pi(S)^{-1}(s_1)\). We perform the reversal \(R(\pi(S), 0, \text{stop})\), which places \(s_1\) at the beginning of \(\pi(S)\) and reverses the order of all elements from index 0 to \(\text{stop}\).

        \item After the reversal, the element \(s_1\) is now at the beginning of \(\pi(S)\). Now the problem reduces to transforming the sub-array starting from \(s_2\) to the end of \(S\) into the remaining portion of \(\pi(S)\). This remaining portion is a permutation of \(S' = (s_2, s_3, \ldots, s_{k+1})\), which is a collection of size \(k\).

        \item By the inductive hypothesis, we can transform \(S'\) into the remaining part of \(\pi(S)\). Since \(S'\) is of size \(k\), by our inductive hypothesis, we know there exists a sequence of sub-array reversals that transforms \(S'\) into the remaining portion of \(\pi(S)\).

        \item Combining these steps, we can transform \(S\) into \(\pi(S)\). First, apply the reversal \(R(\pi(S), 0, \text{stop})\), and then apply the sequence of reversals (from the inductive hypothesis) that transforms \(S'\) into the remaining portion of \(\pi(S)\).
    \end{enumerate}

    \textbf{Conclusion} \\
    Thus, we have proven that any permutation of a finite ordered collection of objects can indeed be created by reversing sub-arrays of the source array.



\end{document}
